\documentclass{article}
\usepackage{amsmath}
\usepackage{amsfonts}
\usepackage{amssymb}
\usepackage{amsthm}
\title{\vspace{-2cm}Gausian Elimination Notes}
\author{Jonathan Math}
\date{}

\begin{document}

\maketitle

\section*{Gausian Elimination}
Gausian elimination is a method for solving systems of linear equations. It involves transforming the system
of equations into an upper triangular form, from which the solutions can be easily obtained through back substitution.

\subsection*{Steps of Gausian Elimination}
\begin{enumerate}
    \item Write the augmented matrix of the system of equations.
    \item Use row operations to transform the matrix into upper triangular form.
    \item Perform back substitution to find the values of the variables.
\end{enumerate}

\subsection*{Row Operations}
The following row operations can be performed on the augmented matrix:
\begin{itemize}
    \item Swap two rows.
    \item Multiply a row by a non-zero scalar.
    \item Add or subtract a multiple of one row to another row.
\end{itemize}

\subsection*{Example}
Consider the system of equations:
\begin{align*}
    2x + 3y &= 5 \\
    4x + y &= 11
\end{align*}
The augmented matrix is:
\[  \begin{bmatrix}
    2 & 3 & | & 5 \\
    4 & 1 & | & 11
\end{bmatrix} \]

We can perform the following row operations to transform it into upper triangular form:
1. Multiply the first row by 2 and subtract it from the second row:
\[  \begin{bmatrix}
    2 & 3 & | & 5 \\
    0 & -5 & | & 1
\end{bmatrix} \]
2. Divide the second row by -5:
\[  \begin{bmatrix}
    2 & 3 & | & 5 \\
    0 & 1 & | & -\frac{1}{5}
\end{bmatrix} \]

Now we can perform back substitution:
From the second row, we have:
\[ y = -\frac{1}{5} \]
Substituting \( y \) into the first row:
\[ 2x + 3\left(-\frac{1}{5}\right) = 5 \]
\[ 2x - \frac{3}{5} = 5 \]
\[ 2x = 5 + \frac{3}{5} = \frac{28}{5} \]
\[ x = \frac{14}{5} \]
Thus, the solution to the system is:
\[ x = \frac{14}{5}, \quad y = -\frac{1}{5} \]


\section*{Reduced Row Echelon Form}
The reduced row echelon form (RREF) of a matrix is a form where:
\begin{itemize}
    \item Each leading entry in a row is 1.
    \item Each leading 1 is the only non-zero entry in its column.
    \item The leading 1 of a row is to the right of the leading 1 of the previous row.
    \item Any rows consisting entirely of zeros are at the bottom of the matrix.
\end{itemize}

\subsection*{Finding RREF}
To find the RREF of a matrix, you can use the following steps:
\begin{enumerate}
    \item Start with the original matrix.
    \item Use row operations to create leading 1s in each row.
    \item Use row operations to ensure that each leading 1 is the only non-zero entry in its column.
    \item Ensure that the leading 1s are in a stair-step pattern from left to right.
    \item Move any rows of zeros to the bottom of the matrix.
\end{enumerate}

\subsection*{Example of RREF}
Consider the matrix:
\[  \begin{bmatrix}
    1 & 2 & 3 \\
    0 & 1 & 4 \\
    0 & 0 & 1
\end{bmatrix} \]
This matrix is already in RREF because:
\begin{itemize}
    \item Each leading entry is 1.
    \item Each leading 1 is the only non-zero entry in its column.
    \item The leading 1s are in a stair-step pattern.
    \item There are no rows of zeros.
\end{itemize}

\section*{Example of RREF Calculation}
Let's find the RREF of the following matrix:
\[  \begin{bmatrix}
    2 & 4 & 6 \\
    1 & 2 & 3 \\
    0 & 1 & 1
\end{bmatrix} \]
1. Start with the original matrix:
\[  \begin{bmatrix}
    2 & 4 & 6 \\
    1 & 2 & 3 \\
    0 & 1 & 1
\end{bmatrix} \]
2. Divide the first row by 2 to create a leading 1:
\[  \begin{bmatrix}
    1 & 2 & 3 \\
    1 & 2 & 3 \\
    0 & 1 & 1
\end{bmatrix} \]
3. Subtract the first row from the second row:
\[  \begin{bmatrix}
    1 & 2 & 3 \\
    0 & 0 & 0 \\
    0 & 1 & 1
\end{bmatrix} \]
4. Now, we can swap the second and third rows to move the non-zero row up:
\[  \begin{bmatrix}
    1 & 2 & 3 \\
    0 & 1 & 1 \\
    0 & 0 & 0
\end{bmatrix} \]
5. Finally, we can subtract 2 times the second row from the first row to eliminate the second column in the first row:
\[  \begin{bmatrix}
    1 & 0 & 1 \\
    0 & 1 & 1 \\
    0 & 0 & 0
\end{bmatrix} \]
This matrix is now in RREF.




\end{document}